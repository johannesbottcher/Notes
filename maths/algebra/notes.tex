\documentclass[10pt,onecolumn]{article}
\usepackage[utf8]{inputenc}
\usepackage[english]{babel}
\usepackage[T1]{fontenc}
\usepackage{lmodern}
\usepackage{multicol}
\usepackage{amsmath}
\usepackage{amssymb}
\usepackage{mathrsfs}
\usepackage{graphics}
\usepackage[margin=1.5cm]{geometry}

\DeclareMathSizes{10}{10}{10}{10}

% --- 
% Beginning of document
% ---
\begin{document}
{\setlength{\parindent}{0cm}
\part{Basic Properties \& Facts}

% --- 
% Arithmetic section
% ---
\section{Arithmetic}
\begin{center}
{\renewcommand{\arraystretch}{2}
\begin{tabular}[c]{| c | c | c | c | c}
\hline
\(\frac{ab + ac}{a} = b + c, a\neq 0 \) & 
\(\frac{a}{b} + \frac{c}{d} = \frac{ad + bc}{bd}\) & 
\(\frac{a - b}{c - d} = \frac{b - a}{d - c}\) &
\(\frac{a + b}{c} = \frac{a}{c} + \frac{b}{c}\) \\
\hline
\(\frac{\frac{a}{b}}{\frac{c}{d}} = \frac{ad}{bc}\) & 
\(\frac{a}{b} - \frac{c}{d} = \frac{ad - bc}{bd}\) & 
\(a \frac{b}{c} = \frac{ab}{c}\) &
\(ab + ac = a(b+c) \) \\
\hline
\end{tabular}}
\end{center}

\subsection{Common arithmetic errors}

\begin{center}
{\renewcommand{\arraystretch}{2}
\begin{tabular}{| c | c | }
\hline
Error & Reason/correction/justification/example \\
\hline
\(\frac{2}{0} \neq 0 \) and \(\frac{2}{0} \neq 2 \) & Division by zero is undefined! \\
\hline
\(\frac{a}{b + c} \neq \frac{a}{b} + \frac{b}{c}\) & \(\frac{1}{2} = \frac{1}{1 + 1} \neq \frac{1}{1} + \frac{1}{1} = 2\)  \\
\hline
\(\frac{1}{x^2 + x^3} \neq x^{-2} + x^{-3}\) & A more complex version of the previous error.  \\
\hline
\(\frac{\not a + bx}{\not a} \neq 1 + bx\)  & \(\frac{a + bx}{a} = \frac{a}{a} + \frac{bx}{a} = 1 + \frac{bx}{a}\)\\
\hline
\(\frac{a}{\frac{b}{c}} \neq \frac{ab}{c}\) & \(\frac{a}{\frac{b}{c}} = \frac{\frac{a}{1}}{\frac{b}{c}} = \frac{a}{1} \frac{c}{b} = \frac{ac}{b}\) \\
\hline
\(\frac{\frac{a}{b}}{c} \neq \frac{ac}{b}\) & \(\frac{\frac{a}{b}}{c} = \frac{\frac{a}{b}}{\frac{c}{1}} = \frac{a}{b} \frac{1}{c} = \frac{a}{bc}\) \\
\hline
\(-a(x-1) \neq -ax - a\) & \(-a(x-1) \neq -ax + a\)  \\
\hline


\end{tabular}}
\end{center}

% --- 
% Exponent section
% ---
\section{Exponent}
\begin{center}
{\renewcommand{\arraystretch}{2}
\begin{tabular}[c]{| c | c | c | c | c |}
\hline
\(a^n a^m = a^{n+m} \) & 
\((a^n)^m = a^{nm} \) & 
\((ab)^n = a^n b^n\) & 
\(a^{-n} = \frac{1}{a^n}\) & 
\((\frac{a}{b})^{-n} = (\frac{b}{a})^{n} = \frac{b^n}{a^n} \) \\
\hline
\(\frac{a^n}{a^m} = a^{n-m} = \frac{1}{a^{m-n}}\) & 
\(a^0 = 1, a \neq 0 \) & 
\((\frac{a}{b})^n = \frac{a^n}{b^n}\) & 
\(\frac{1}{a^{-n}} = a^n\) & 
\(a^\frac{n}{m} = (a^\frac{1}{m})^n = (a^n)^\frac{1}{m}\) \\
\hline
\end{tabular}}
\end{center}

\subsection{Common exponent errors}

\begin{center}
{\renewcommand{\arraystretch}{2}
\begin{tabular}{| c | l | }
\hline
Error & Reason/correction/justification/example \\
\hline
\(-3^2 \neq 9\) & \(-3^2 = -9\), \((-3^2) = 9\) Watch parenthesis !  \\
\hline
\((x^2)^3 \neq x^5\) & \((x^2)^3 = x^6\)   \\
\hline
\((x + a)^2 \neq x^2 + a^2\) & \((x + a)^2 = (x + a)(x + a) = x^2 + 2ax + a^2 \) \\
\hline
\(2(x+1)^2 \neq (2x + 2)^2 \) & \(2(x + 1)^2 = 2(x^2 + 2x + 1) = 2x^2 + 4x + 2\), \\ 
& \((2x^2 + 2)^2 = 4x^2 +8x + 4 \) \\
\hline
\((2x + 2)^2 \neq 2(x + 1)^2\) & No factor out a constant if there is a power on the parenthesis! \\
\hline
\((x + a)^n \neq x^n + a^n\) and \(\sqrt[n]{x + a} \neq \sqrt[n]{x} + \sqrt[n]{a}\) & More general versions of previous three errors \\
\hline
\end{tabular}}
\end{center}


% --- 
% Radicals section
% ---
\section{Radicals}
\begin{center}
{\renewcommand{\arraystretch}{2}
\begin{tabular}[c]{| c | c | c |}
\hline
\(\sqrt[n]{a^m} = a^{\frac{m}{n}}\) &
\(\sqrt[m]{\sqrt[n]{a}} = \sqrt[mn]{a}\) &
\(\sqrt[n]{a^n} = a, \text{if n is odd} \) \\
\hline
\(\sqrt[n]{a^n} = |a|, \text{if n is even} \) &
\(\sqrt[n]{ab} = \sqrt[n]{a}\sqrt[n]{b}\) &
\(\sqrt[n]{\frac{a}{b}} = \frac{\sqrt[n]{a}}{\sqrt[n]{b}}\) \\
\hline
\end{tabular}}
\end{center}

\subsection{Square Root property}
If \( x^2 = p \) then \( x = \pm \sqrt{p}\)

\subsection{Common radicals errors}

\begin{center}
{\renewcommand{\arraystretch}{2}
\begin{tabular}{| c | c | }
\hline
Error & Reason/correction/justification/example \\
\hline
\(\sqrt{x^2 + a^2} \neq x + a \) & \(5 = \sqrt{25} = \sqrt{3^2 + 4^2} \neq \sqrt{3^2} + \sqrt{4^2} = 3 + 4 = 7 \) \\
\hline
\(\sqrt{x + a} \neq \sqrt{x} + \sqrt{a} \) & See previous error \\
\hline
\(\sqrt{-x^2 + a^2} \neq -\sqrt{x^2 + a^2} \) & \(\sqrt{-x^2 + a^2} = (-x^2 + a^2)^\frac{1}{2}\) \\
\hline
\end{tabular}}
\end{center}

% --- 
% Inequalities section
% ---
\section{Inequalities}
If a < b then a + c < b + c and a - c < b - c \\\\
If a < b and c > 0 then ac < bc and \(\frac{a}{c} < \frac{b}{c}\) \\\\
If a < b and c < 0 then ac > bc and \(\frac{a}{c} > \frac{b}{c}\) \\\\
\text{When both sides a multiplied by a negative, the inequality's side must be changed} \\
\begin{tabular}{ccc}
-a & < & b \\
a & > & -b
\end{tabular} \\

\text{A constant/variable must be moved on all sides of an inequality} \\
\begin{tabular}{ccccc}
a & < & x + b & < & c \\
a - b & < & x & < & c - b \\ 
or \\
a & < & x*b & < & c \\
\(\frac{a}{b} \) & < & x & < & \(\frac{c}{b} \)\\

\end{tabular}

% --- 
% Absolute value section
% ---
\section{Absolute value}
\begin{center}
{\renewcommand{\arraystretch}{2}
\begin{tabular}{| c | c | c |}
\hline
\(\left|a\right| = a \), if \(a \ge 0\), -a if \(a < 0\) &
\(\left|a\right| \ge 0 \) &
\(\left|ab\right| = \left|a\right|\left|b\right| \) \\
\hline
\(\left|a+b\right| \le \left|a\right| + \left|b\right|\) &
\(\left|-a\right| = \left|a\right| \) &
\(\left|\frac{a}{b}\right| = \frac{\left|a\right|}{\left|b\right|} \) \\
\hline
\end{tabular}}
\end{center}

% --- 
% Absolute value section
% ---
\section{Absolute value and inequalities}
\begin{center}
{\renewcommand{\arraystretch}{2}
\begin{tabular}{| c | c |}
\hline
\(\left|p\right| = b \) & \(p = -b \) or \( p = b \) \\
\hline
\(\left|p\right| < b \) & \(-b < p < b \) \\
\hline
\(\left|p\right| > b \) & \(p < -b \) or \( p > b \) \\
\hline
\end{tabular}}
\end{center}

% --- 
% Complex numbers section
% ---
\section{Complex numbers}
\begin{center}
{\renewcommand{\arraystretch}{2}
\begin{tabular}{| c | c |}
\hline
\(\sqrt{-a} = i\sqrt{a}, a \ge 0 \) &
\((a + bi) + (c + di) = a+ c + (b + d)i\) \\
\hline
\((a + bi) - (c + di) = a - c + (b - d)i \) &
\((a + bi)(c - di) = ac - bd + (ad + bc)i\) \\
\hline
\((a + bi)(a - bi) = a^2 + b^2 \) &
\(\left|a + bi\right| = \sqrt{a^2 + b^2} \) Complex Modulus \\
\hline
\(\bar{(a + bi)}(a + bi) = \left|a + bi\right|^2 \) &
\(\bar{(a + bi)} = a - bi \) Complex Conjugate \\
\hline
\end{tabular}}
\end{center}

\subsection{Main values of \(i \)}
\begin{tabular}{c c c}
\(i\) & = & \(\sqrt{-1} \) \\
\(i^2 \) & = & \(-1\) \\
\(i^3 \) & = & \(-i\) \\
\(i^4 \) & = & \(1\) \\
\end{tabular}

% --- 
% Logarithms and Log properties section
% ---
\section{Logarithms and Log properties}
\subsection{Definition and special Logarithms}
\(y = log_b x\) is equivalent to \(x = b^y\) \\
\(ln x = log_e x \) (Natural Log) \\
\(log x = log_{10} x \) (Common Log) \\
The domain of \(log_b x\) is \(x > 0\) 

\subsection{Logarithms properties}
\begin{center}
{\renewcommand{\arraystretch}{2}
\begin{tabular}{| c | c | c | c |}
\hline
\(log_b b = 1 \) &
\(log_b 1 = 0 \) &
\(log_b b^x = x \) &
\(b^{log_b x} = x \) \\
\hline
\(log_b (x^r) = r * log_b x \) &
\(log_b (xy) = log_b x + log_b y \) &
\(log_b (\frac{x}{y}) = log_b x - log_b y\) &
\(log_b x = \frac{log_d(x)}{log_d(b)} \) \\
\hline
\end{tabular}}
\end{center}

% --- 
% Factoring and Solving section
% ---
\section{Factoring and Expanding}
\begin{tabular}{l l l}
\(x^2 - a^2 \) & = & \( (x+a)(x-a)\) \\
\((x+a)^2 \) & = & \( x^2 + 2ax + a^2 \) \\
\((x-a)^2 \) & = & \( x^2 - 2ax + a^2 \) \\
\((x+a)(x+b) \) & = & \( x^2 + (a+b)x + ab \) \\
\((x+a)^3 \) & = & \( x^3 + 3ax^2 + 3a^{2}x + a^3\) \\
\((x-a)^3 \) & = & \( x^3 - 3ax^2 + 3a^{2}x - a^3\) \\
\(x^3 + a^3 \) & = & \( (x+a)(x^2 - ax + a^2)\) \\
\(x^3 - a^3 \) & = & \( (x-a)(x^2 + ax + a^2)\) \\
\(x^{2n} - a^{2n} \) & = & \( (x^n - a^n)(x^n + a^n)\) \\
\end{tabular}}\\

% --- 
% Quadratic formula section
% ---
\section{Quadratic formula}
Find the roots of a quadratic equation : \\
\( ax^2 + bx + c = 0; a \neq 0 \) \\
\( x = \frac{-b \pm \sqrt{b^2 - 4ac}}{2a} \)

% --- 
% Completing the square section
% ---
\section{Completing the square}
\begin{center}
Solve \(2x^2 - 6x - 10 = 0 \) \\
{\renewcommand{\arraystretch}{2}
\begin{tabular}{l l}
(1) Divide the coefficient of the \(x^2\) & (4) Factor the left side \\
\multicolumn{1}{c}{\(x^2 -3x - 5\)} & \multicolumn{1}{c}{\((x - \frac{3}{2})^2 = \frac{29}{4}\)} \\
(2) Move the constant to the other side & (5) Use square root property \\
\multicolumn{1}{c}{\(x^2 - 3x = 5\)} & \multicolumn{1}{c}{\(x - \frac{3}{2} = \pm \sqrt{\frac{29}{4}}\)} \\
(3) Take half the coef. of x, square it and add it to both sides & (6) Solve for x \\
\multicolumn{1}{c}{\(x^2 - 3x + (\frac{-3}{2})^2 = 5 + (\frac{-3}{2})^2 \)} & \multicolumn{1}{c}{\(x = \frac{3}{2} \pm \frac{\sqrt{29}}{2}\)} 
\end{tabular}}
\end{center}
\pagebreak

\part{Functions and Graphs}

% --- 
% Distance formula section
% ---
\section{Distance formula}
\(If P_1 = (x_1, y_1) \) and \(P_2 = (x_2, y_2)\) are two points. The distance between them is equal to : \\
\begin{center}
\(d(P_1, P_2) = \sqrt{(x_2 - x_1)^2 + (y_2 - y_1)^2} \)
\end{center}

% --- 
% Constant Function
% ---
\section{Constant function}
\(y = a\) or \(f(x) = a \) \\ 
Graph is a horizontal line passing through the point (0,a). \\
Domain \(= \mathbb{R}\)

\section{Linear function}
\(f(x) = mx + b \) \\
Graph is a line with point (0,b) and slope m. \\\\
Slope \\
\(\sqrt{\frac{y_2 - y_1}{x_2 - x_1}} = \frac{rise}{run}\) \\\\
Slope - intercept form \\
The equation of the line with slope m and y-intercept (0, b) is 
\(y = mx + b \) \\\\
Point - Slope from \\
The equation of the line with slope m and passing through the point \((x_1, y_1)\) is \\
\(y = y_1 + m(x - x_1) \)

\section{Parabola/Quadratic function}
\(f(x) = a(x - h)^2 \) \\\\
The graph is a parabola that opens up if \(a > 0\) or down if \(a < 0\) and has a vertex at \\
\(\frac{-b}{2a}, f - \frac{b}{2a} \) \\
\(g(y) or x = ay^2 + by + c \) \\
This graph is a parabola that opens right if \(a > 0\) or left  if \(a < 0\) and has a vertex at \\
\(\frac{-b}{2a}, g - \frac{b}{2a} \)

\section{Circle}
\((x - h)^2 + (y - k)^2 = r^2  \) \\
The graph is a circle with radius r and center (h, k).

\section{Ellipse}
\(\frac{(x - h)^2}{a^2} + \frac{(y - k)^2}{b^2} = 1\)  \\
Graph is an ellipse with center (h, k) with vertices a units right/left from the center and vertices b units up/down from the center.

\section{Hyperbola}
\(\frac{(x - h)^2}{a^2} - \frac{(y - k)^2}{b^2} = 1\)  \\
Graph is a hyperbola that opens left and right, has a center at (h, k), vertices a units left/right of center and asymptotes that pass through center with slope \(\pm \frac{b}{a}\) \\
\(\frac{(y - k)^2}{b^2} - \frac{(x - h)^2}{a^2} = 1\)  \\
Graph is a hyperbola that opens up and down, has a center at (h, k), vertices b units up/down of center and asymptotes that pass through center with slope \(\pm \frac{b}{a}\) \\

% --- 
% End of document
% ---
\end{document}
