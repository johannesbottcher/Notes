\documentclass[10pt,onecolumn]{article}
\usepackage[utf8]{inputenc}
\usepackage[english]{babel}
\usepackage[T1]{fontenc}
\usepackage{lmodern}
\usepackage{multicol}
\usepackage{amsmath}
\usepackage{amssymb}
\usepackage{mathrsfs}
\usepackage[margin=1cm]{geometry}

\DeclareMathSizes{10}{10}{10}{10}

% --- 
% Beginning of document
% ---
\begin{document}
{\setlength{\parindent}{0cm}
Geometry

\part{Formulas}

% --- 
% Angles
% ---
\section{Angles }
The sum of the measures of the interior angles of a convex polygon with n sides is (n-2)180 \\
Examples : \\
Triangle : (3 - 2)180 = \(180^\circ\) \\
Quadrilateral : (4 - 2)180  = \(360^\circ\) \\
Hexagon : (6 - 2)180 =  = \(720^\circ\) \\

Regular polygon (polygon with sides of equal length and angles of equal measure) with n sides : \\
\(\frac{(n - 2) 180^\circ}{n}\) \\

If two parallel lines are crossed by another line : \\
- Corresponding angles are equal \\
- Alternate Interior angles are equal \\
- Alsternate Exterior angles are equal \\
- Samde side interior angles are supplementary. \\

A linear pair is two adjacent angles that form a straight angle. \\
A vertical angle is formed at the intersection of two straight lines. \\
% --- 
% Transformation
% ---
\section{Transformation}
Rotation : \\
Rotation by a certain angle around a point. \\

Reflection : \\
A reflection is a flip over a mirror line.\\
When the mirror line is the x-axis, P(x, y) \(\rightarrow\) P(x, -y). \\
When the mirror line is the y-axis, P(x, y) \(\rightarrow\) P(-x, y). \\

Translation : \\
Every point of the shape must move the same distance and in the same direction.

If one shape becomes another with rotate, reflection, translation : the shapes are congruent. \\

Resizing : \\

If one shape becomes another using a resize (and other transformations) : the shapes are similar. \\

% --- 
% Triangle
% ---
\section{Triangle }
Triangle ABC \\
Scalene  : \(\overline{AB} \neq \overline{BC} \neq \overline{AC} \), no congruent sides \\
Isocele  : \(\overline{AB} = \overline{BC} \neq \overline{AC} \) 2 congruent sides,  \(\angle BCA = \angle BAC\) \\
Equilateral : \(\overline{AB} = \overline{BC} = \overline{AC} \) 3 congruent sides, all angles equal \(60^\circ \rightarrow\) equiangular \\
Right triangle :  one angle is \(90^\circ\) \\
Area = \(\frac{bh}{2}\) \\
Exterior angle of triangle equals the sum of the 2 non-adjacent interior angles. \\
Mid-segment of a triangle is parallel of the third side and half the length of the third side. \\
Triangles are similar when their corresponding sides are in proportion. \\
The sum of the lengths of any two sides of a triangle is greater than the length of the third side. \\
Longest side of a triangle is opposite the largest angle. \\
Exterior angle of a triangle is greater than either of the two non-adjacent interior angles. \\

% --- 
% Rectangle
% ---
\section{Rectangle }

Sum of angles : \(360^\circ \) \\
Area : \(lw\)

% --- 
% Trapezoid
% ---
\section{Trapezoid }
Sum of angles : \(360^\circ \) \\
Area : \(\frac{(b1 + b2)h}{2}\) \\

% --- 
% Parallelogram
% ---
\section{Parallelogram }
Sum of angles : \(360^\circ \) \\
Area : \(bh\)

% --- 
% Circle
% ---
\section{Circle }
Radius = r \\
Area : \(\pi r^2\) \\
Circumference : \(2\pi r\) \\
The ratio between the arc's central angle \(\theta\) and \(2 \pi \) radians is equal to the ratio between the arc length and the circle's circumference. \\

Circle Angles : \\
Central angle = arc \\
Inscribed angle = half arc \\
Angle by tangent/chord = half arc \\
Angle formed by 2 chords = half the sum of arcs \\
Angle formed by two tangents, or 2 secants, or a tangent/secant = half the difference of arcs \\

Equation : \\
- of circle center at origin : \(x^2 + y^2 = r^2\) \\
- or circle not at origin : \((x - h)^2 + (y - k)^2 = r^2\) where (h, k) is the center and r the radius. \\

% --- 
% Rectangular Prism
% ---
\section{Rectangular Prism}
Volume = \(lwh\) \\
Surface Area : \(2lw + 2hw + 2lh\) \\

% --- 
%  General Prism
% ---
\section{General Prism}
V = Bh = area of base x height \\
SA : sum of the area of the faces \\

% --- 
% Right Circular Cylinder
% ---
\section{Right Circular Cylinder}
V = \(\pi r^2 h\) = area of base x height \\
SA =\(2 \pi rh + 2 \pi r^2\) = 2 times area of base + circumference x height \\

% --- 
% Square Pyramid
% ---
\section{Square Pyramid}
V = \(\frac{Bh}{3} = \frac{1}{3}\) x area of base x height \\
SA = B + \(\frac{1}{2}\) Pl = area of base + (\(\frac{1}{2}\) x perimeter of base x slant height) \\

% --- 
% Right Circular Cone
% ---
\section{Right Circular Cone}
V = \(\frac{\pi r^2 h}{3}\) \\
SA = B + \(\frac{1}{2}\) Cl = area of base + (\(\frac{1}{2}\) x circumference x slant height) \\

% --- 
% Sphere
% ---
\section{Sphere}
V = \(\frac{4\pi r^3}{3}\) \\
SA = \(4\pi r^2\) \\

% --- 
% Regular Solids
% ---
\section{Regular Solids}
Tetrahedron : 4 faces \\
Cube : 6 faces \\
Octahedron : 8 faces \\
Dodecahedron : 12 faces \\
Icosahedron : 20 faces \\

\part{Coordinate}
\(P_1 = (x_1, y_1), P_2 = (x_2, y_2)\) are two points in the plane : \\
slope = \(\frac{y_2 - y_1}{x_2 - x_1}\) where \(x_2 \neq x_1\) \\
midpoint = \(\left(\frac{\frac{x_1 + x_2}{2}}{\frac{y_1 + y_2}{2}}\right)\) \\
distance = \(\sqrt{(x_2 - x_1)^2 + (y_2 - y_1)^2}\) \\

% --- 
% End of document
% ---
\end{document}
